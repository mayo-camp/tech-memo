\usepackage{tcolorbox}
\tcbuselibrary{listingsutf8}
\usepackage{needspace}
\usepackage{xcolor}
\usepackage{float}
\usepackage{array}

\raggedbottom % 不自然な空白改ページを減らす

% 表を左寄せにする設定
\makeatletter
\renewenvironment{table}[1][htbp]{%
  \@float{table}[H]%
}{%
  \end@float
}
\makeatother

% コードブロック用のスタイル(GitHub風, モノクロ印刷対応)
\newtcolorbox{codeblock}{
  breakable,
  colback=gray!15,
  colframe=gray!50,
  boxrule=0.5pt,
  arc=1.5mm,
  left=6pt, right=6pt, top=6pt, bottom=6pt,
  listing only,
  listing options={
    basicstyle=\ttfamily\footnotesize,
    breaklines=true,
    breakatwhitespace=true,
    columns=flexible,
  }
}

% インラインコード用のスタイル(GitHub風)
\definecolor{inlinegray}{gray}{0.9}
\newcommand{\inlinecode}[1]{%
  \begingroup
  \setlength{\fboxsep}{1pt}%
  \colorbox{inlinegray}{\texttt{#1}}%
  \endgroup
}

% 見出し前に最低限確保する行数
\newcommand{\sectionbreak}{\Needspace{3\baselineskip}}
