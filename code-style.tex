% GitHub風 コードブロック & インラインコード

\usepackage{tcolorbox}
\tcbuselibrary{listings}
\tcbuselibrary{breakable}

% カラー設定(GitHub風)
\definecolor{gh-bg}{HTML}{F6F8FA}     % 背景色
\definecolor{gh-border}{HTML}{D0D7DE} % 枠線色

% コードブロック環境
\newtcblisting{codeblock}{
  listing only,
  breakable,
  colback=gh-bg,
  colframe=gh-border,
  boxrule=0.3pt,
  arc=2pt,
  outer arc=2pt,
  left=5pt,
  right=5pt,
  top=5pt,
  bottom=5pt,
  listing options={
    basicstyle=\ttfamily\footnotesize,
    columns=fullflexible,
    breaklines=true,
    keepspaces=true,
    showstringspaces=false
  }
}

% インラインコード (`code`)
\newtcbox{\icode}{
  on line,
  colback=gh-bg,
  colframe=gh-border,
  boxrule=0.3pt,
  arc=2pt,
  outer arc=2pt,
  left=2pt,
  right=2pt,
  top=1pt,
  bottom=1pt,
  boxsep=0pt,
  fontupper=\ttfamily\footnotesize
}

% 表を GitHub風(横線のみ、左寄せ)
\usepackage{array}
\usepackage{booktabs}
\usepackage{longtable}

\renewcommand{\arraystretch}{1.2} % 行の高さ調整
\setlength{\tabcolsep}{6pt}       % セル左右余白
\LTleft=0pt
\LTright=0pt
